\newcommand{\In}{\ensuremath{\mathtt{in}\ }}
\newcommand{\Register}{\ensuremath{\mathtt{register}\ }}
\newcommand{\Select}{\ensuremath{\mathtt{select}\ }}
\newcommand{\To}{\ensuremath{\mathtt{to}\ }}
\newcommand{\From}{\ensuremath{\mathtt{from}\ }}
\newcommand{\New}{\ensuremath{\mathtt{new}\ }}
\newcommand{\Selector}{\ensuremath{\mathtt{selector}\ }}

\newcommand{\newsel}[2]{\New \Selector[#1]\ #2\ \In}
\newcommand{\register}[2]{\Register #1\ \From\ #2\ \In}
\newcommand{\select}[2]{\Select #1\ \From\ #2\ \In}

\newcommand{\shqueue}[2]{#1[#2]}

\newcommand{\es}{\ensuremath{\epsilon}}

\newcommand{\Lock}{\ensuremath{\mathbf{Lock}}}

\newcommand{\out}[2]{\ensuremath{#1!\langle #2 \rangle;}}
\newcommand{\inp}[2]{\ensuremath{#1?(#2);}}
\newcommand{\Par}{|}


\section{Session type Implementation}
\label{sec:scribble}

We use the selector construct~\cite{citation_needed}
to describe a lock process that registers two processes that
want to acquire the lock.

\begin{eqnarray*}
	\Lock &=& \newsel{S}{r} \inp{a}{x} \\
	& & \register{x}{r} \inp{a}{y} \\
	& & \register{y}{r} \\
	& & \mu X. \inp{x}{z} \inp{x}{w} X
\end{eqnarray*}

with
\[
  S = \mu X. !\langle \lock \rangle; ?(\unlock); X
\]


We then define processes $A$ and $B$ that interact with the lock 
\begin{eqnarray*}
	A &=& \out{a}{s_1} \out{s_1}{\lock} \dots \mathsf{critical\ section} \dots \out{s_1}{\unlock} \\
	B &=& \out{a}{s_2} \out{s_2}{\lock} \dots \mathsf{critical\ section} \dots \out{s_2}{\unlock} \\
\end{eqnarray*}

The whole system would be
\[
	\Lock \Par A \Par B \Par \shqueue{a}{\es}
\]
