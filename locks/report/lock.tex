\section{Description}

In this report we describe a session type implementation
of a {\em lock}. A lock is basic construct in concurrent
programming.
They are typicaly used 
to control and co-ordinate the asynchronous and reactive
nature of concurrent programs.
The semantics for locks is fundamental for concurrent
programs. In fact, without such semantics the implementation
of basic concurrent structures (i.e.~mutual exclusion patterns)
is impossible.

The semantics for locks can be easily encoded in
message passing semantics. The purpose of this
usecase is to demonstrate a message passing encoding
of a lock which is session typed. This is to
identify the session type that specifies this
fundamental concurrency structure.

A session type description of a lock will further
ensure the correctness of different mutual exclusion algorithms.

\newcommand{\lock}{\ensuremath{\mathsf{lock}}\xspace}
\newcommand{\unlock}{\ensuremath{\mathsf{unlock}}\xspace}

\subsection{Lock Semantics}

We informally describe the semantics of a lock structure.

\begin{itemize}
	\item	A lock may be abstracted as a state machine.

	\item	The initial state is the {\em unclocked} state.

	\item	Any process registered to the lock may interact with the
			lock via messages \lock and \unlock.

	\item	A process $A$ may send the message \lock and interact with a lock in the unlock state.
			The result of the interaction will force the lock to transition to state {\em lock $A$}.

	\item	At a lock $A$ state a lock may only interact with an \unlock message from process $A$
			to transit back to the unlock state.

	\item	
\end{itemize}

\subsection{Session Types}

The main challenge regarding a lock implementation
in session types, is the fact that any process may
interact with the process at the same time, creating 
a race condition.

Standard binary session types and multiparty session types
do not support race conditions. The description of
protocols in session types assume deterministic interaction
between roles. If we were to allow race conditions to standard
session types we would lose vital progress properties, since
roles that {\em lose the race} would remain deadlocked.

We demonstrate the above intuition using an example:

\newcommand{\val}[2]{\ensuremath{#1 \rightarrow #2;} }
\newcommand{\tinact}{\ensuremath{\mathbf{end}}}
\newcommand{\tout}[1]{\ensuremath{[#1]!;}}
\newcommand{\tinp}[1]{\ensuremath{[#1]?;}}

If we use a non-deterministic protocol ($+$ is non-deterministic internal choice)
to describe a lock:
\[
	\val{A}{L}; \tinact + \val{B}{L} \tinact
\]
we can see that in the local projection semantics:
\begin{eqnarray*}
	&A:& \tout{L} \tinact\\
	&B:& \tout{L} \tinact\\
	&L:& \tinp{A} \tinact + \tinp{B} \tinact
\end{eqnarray*}
Either role $A$ or role $B$ will remain at a deadlock state
if the other role interacts with the lock.


To overcome such problems Kouzapas et.~al.~\cite{KouzapasYHH13}
developed a binary session type system that supports asynchronous
semantics and runtime message and type inspection able to describe
reactive communication patterns such as the above. The theory
was implemented the Eventful Session Java (ESJ) framework
by Hu et.~al.~\cite{event}.
We use the theory for eventful session types and the ESJ framework
to implement the requirements of this usecase.
