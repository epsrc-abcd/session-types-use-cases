\section{Description}

In this report we describe the implementation of a
concurrent algorithm that computes the Fibonacci
sequence. The communication between the concurrent 
programs is described using global types.
The implementation is done on the \Mungo framework.
The Fibonacci sequence is a sequence of numbers;
each number in the sequence is equal to the sum
of the two previous number in the sequence:
%
\begin{eqnarray*}
	T_0 &=& 0 \\
	T_1 &=& 1\\
	T_n &=& T_{n-1} + T_{n-2}
\end{eqnarray*}

\subsection{Algorithm}
\label{subsec:algorithm}
The algorithm requires two threads that communicate
over a channel with a socket interface.
The communication of the two threads is described by a
two role global protocol, thus each thread implements
a role of the protocol.

The initial state of the Fibonacci program is:
%
\begin{itemize}
	\item	The program creates two threads; the first thread
			executes the code of role \A and the second thread
			executes the code of role \B.

	\item	Roles \A and \B create, respectively,
			the two endpoints of a channel. Each endpoint
			is an object with a socket interface.

	\item	Role \A takes as input the sequence number
			of the Fibonacci number to be computed.

	\item	Role \A initialises its Fibonacci number
			with the
			the zero Fibonacci number $T_0 = 0$ and 
			role \B initialises its Fibonacci number with
			the first Fibonacci number $T_1 = 1$.
\end{itemize}

Both roles follow a global communication protocol
%that is described in Scribble.
Each of the roles implemented their own contribution
of the global communication protocol.

The protocol dictates that \A decides whether to iterate over a computation
state that sends a message to \B to continue computation or send a message to \B
to end the protocol.
In the former case \A sends its Fibonacci number to \B, \B adds
the received number to its Fibonacci number to 
compute the next Fibonacci number and sends the result back to \A.
Role \A then decides whether to continue computation or stop. The decision
is based on the number of iterations done, so to compute the desired Fibonacci number.
Note that each computation performed in either role \A or role \B, computes a
Fibonacci number.


